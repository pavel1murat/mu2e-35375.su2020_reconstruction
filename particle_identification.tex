%%% Local Variables:
%%% mode: latex
%%% TeX-master: "mu2e-36575"
%%% End:

%%%%%%%%%%%%%%%%%%%%%%%%%%%%%%%%%%%%%%%%%%%%%%%%%%%%%%%%%%%%%%%%%%%%%%%%%%%%%%
\section{Particle Identification}

Datasets used for MVA training : {\bf ele00s61b0} and {\bf mumi0s61b0}


\subsection {Event selection}

After the reconstructed calorimeter cluster is included into the track fit, it biases the fit results.

Something ( a bug?) in the track fit pulls the track timing to the cluster and biases
the reconstructed track timing.

There  is a correlation between the cluster timing residual and the reconstructed Z-coordinate of the
``calorimeter cluster hit'', which the fitter varies in order to minimize the timing residual.

As inclusion of the cluster into the track fit stabilizes the fit and improves its efficiency,
we use track fits with the calorimeter cluster included.

\subsection {Training}

MVA classifiers were trained only for DAR resolver, however based on the nature of 
the variables used, we expect the trained MVA to perform equally well for PAR tracks.

Ran electron and muon reconstruction on the same event, MVA training used inputs from both

For training: events which pass electron selection cuts

Variables and their ranking:

We didn't consider pathological cases with muons passing electron reconstruction, but failing
the muon one, and assume the probability of that to be negligibly small.

